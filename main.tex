%%%%%%%%%%%%%%%%%%%%%%%%%%%%%%%%%%%%%%%%%%%%%%%%%%%%%
%% File: main.tex
%% Author: Evangelos Stamos (estamos@e-ce.uth.gr)
%% Last update: February, 2021
%% Description: Provides an example of a Diploma Thesis 
%% using the ntua-thesis pdfLaTeX class.
%%
%% Character encoding: UTF-8
%%%%%%%%%%%%%%%%%%%%%%%%%%%%%%%%%%%%%%%%%%%%%%%%%%%%%
%
%
%%%%%%
% 1. use the "modern" or "classic" option to switch between 
% a modern or classic font, respectively.
%
% 2. add/remove the "hyperref" option to enable/disable hyperlinks:
% (remember to remove auxiliary files after adding/removing 
% the "hyperref" option).
%
% 3. add/remove the "printer" option to typeset a printer-friendly 
% (grayscale)/color version of the thesis.
%
% 4. use the "watermark" option to indicate that this is not an actual
% thesis.
%
% 5. use the "histinit" option to enable "historiated initials".
% (If used, all chapter initials declared by the \InitialCharacter{}
% macro are enlarged. If omitted, arguments of \InitialCharacter{}
% are typeset as normal text.)
%
% 6. use the "plain" option to disable tikz graphics in title page
% and part/chapter headers (might help to avoid compilation timeouts).
% Note that "plain" disables CD label and CD cover creation.
%
% 7. use the "noindex" option to (hopefully) avoid compilation timeouts
% when compiling online (disables index generation - note that "\indexGR",
% "\indexEN" invocations need not be removed when toggling this option).
%
% 8. activate the "newlogo" option to use the new official Logo.
%
%%%%%%%%%%%%%%%%%%%%%%%%%%%%%%%%%%%%%%%%%%%%%%%%%%%%%%%%%%%%%%%%%%%%%%%%%%%%%%%
%
\documentclass[modern,hyperref,watermark,histinit,noindex,plain,newlogo]{ntua-thesis}
%
%%%%%%%%%%%%%%%%%%%%%%%%%%%%%%%%%%%%%%%%%%%%%%%%%%%%%%%%%%%%%%%%%%%%%%%%%%%%%%%
%
%
%%%%%%%%%%%%%%%%%%%%%%%%%%%%%%%%%%%%%%%%%%%%%%%%%%%%
%% THESIS INFO 
%%%%%%%%%%%%%%%%%%%%%%%%%%%%%%%%%%%%%%%%%%%%%%%%%%%%
%
% ΤΙΤΛΟΣ ΔΙΠΛΩΜΑΤΙΚΗΣ ΕΡΓΑΣΙΑΣ 
%
% Για εξαναγκασμένες αλλαγές γραμμής χρησιμοποιήστε "\\".
% Αν οι αλλαγές γραμμής πρέπει να είναι διαφορετικές στο εξώφυλλο σε σχέση 
% με το εσώφυλλο (σελ. 3), επαναλάβετε τον τίτλο του εξωφύλλου με τις 
% επιθυμητές αλλαγές γραμμής ως προαιρετικό όρισμα της εντολής \title.
%
% Παραδείγματα:
% 1. Όμοιος τίτλος σε εξώφυλλο και εσώφυλλο, με αυτόματες αλλαγές γραμμής:
%	    \title{Πρότυπο Σύστημα Ομότιμων Κόμβων Βασισμένο σε Σχήματα \en{RDF}}
% 2. Όμοιος τίτλος σε εξώφυλλο και εσώφυλλο, με αλλαγή γραμμής μετά τη λέξη
% "Σύστημα":
%	    \title{Πρότυπο Σύστημα \\ Ομότιμων Κόμβων Βασισμένο σε Σχήματα \en{RDF}}
% 3. Διαφορετικές αλλαγές γραμμής σε εξώφυλλο και εσώφυλλο. Στο εξώφυλλο 
% έχουμε αλλαγή γραμμής μετά τη λέξη "Σύστημα", ενώ στο εσώφυλλο η αλλαγή
% γραμμής ακολουθεί τη λέξη "Ομότιμων":
%	    \title[Πρότυπο Σύστημα \\ Ομότιμων Κόμβων Βασισμένο %
%           σε Σχήματα \en{RDF}]% (προαιρετικό όρισμα)
%           {Πρότυπο Σύστημα Ομότιμων \\ Κόμβων Βασισμένο σε %
%           Σχήματα \en{RDF}}% (υποχρεωτικό όρισμα)
%
	\title{Thesis Title}
%%
%
%% -------------------------------------------------------------------
%% ΥΠΟΤΙΤΛΟΣ ΔΙΠΛΩΜΑΤΙΚΗΣ ΕΡΓΑΣΙΑΣ (προαιρετικός)
%
% Αν δεν υπάρχει υπότιτλος, τοποθετήστε τον χαρακτήρα του σχολίου "%"
% πριν από την εντολή \subtitle, ή αφήστε κενό το όρισμα της εντολής.
%
% Παράδειγμα:
%%	\subtitle{Μελέτη και υλοποίηση}
	\subtitle{Thesis subtitle}
%
%% -------------------------------------------------------------------
%% ΤΟΥ/ΤΗΣ/ΤΩΝ
%
% "του" ή "της" ή "των", ανάλογα με το φύλο/αριθμό του σπουδαστή ή 
% των σπουδαστών
% Παράδειγμα:
%	\toutis{του}
	\toutis{\en{of}}
%
%% -------------------------------------------------------------------
%% ΟΝΟΜΑΤΕΠΩΝΥΜΟ ΣΠΟΥΔΑΣΤΗ ΣΤΑ ΕΛΛΗΝΙΚΑ (ΚΕΦΑΛΑΙΑ, ΓΕΝΙΚΗ ΠΤΩΣΗ)
%
% Για περισσότερους του ενός σπουδαστές, διαχωρίστε με ",".
% Παράδειγμα:
%	\authorNameCapitalGR{ΚΩΝΣΤΑΝΤΙΝΟΥ Δ. ΔΗΜΗΤΡΙΟΥ, ΓΕΩΡΓΙΟΥ Π. ΠΑΝΑΓΑΚΗ}
%	\authorNameCapitalGR{ΣΤΑΜΟΥ Φ. ΕΥΑΓΓΕΛΟΥ}
%
%% -------------------------------------------------------------------
%% ΟΝΟΜΑΤΕΠΩΝΥΜΟ ΣΠΟΥΔΑΣΤΗ ΣΤΗ ΛΑΤΙΝΙΚΗ ΜΟΡΦΗ (ΠΕΖΑ)
%
% Δηλώστε εδώ τυχόν ονοματεπώνυμα στη λατινική μορφή, αλλιώς αφήστε
% κενό το όρισμα.
% Για περισσότερους του ενός σπουδαστές, διαχωρίστε με ",".
% Παράδειγμα:
%	\authorNameEN{Albert Einstein, George W. Bush} 
	\authorNameEN{Evangelos F. Stamos} 
%
%% -------------------------------------------------------------------
%% ΟΝΟΜΑΤΕΠΩΝΥΜΟ ΣΠΟΥΔΑΣΤΗ ΣΤΑ ΕΛΛΗΝΙΚΑ (ΠΕΖΑ, ΟΝΟΜΑΣΤΙΚΗ ΠΤΩΣΗ)
%
% Για περισσότερους του ενός σπουδαστές, διαχωρίστε με ",".
% Αν τα ονοματεπώνυμα όλων των σπουδαστών είναι σε λατινική μορφή,
% αφήστε κενό το όρισμα.
% Παράδειγμα:
%	\authorNameGR{Κωνσταντίνος Δημητρίου, Γεώργιος Παναγάκης}
%	\authorNameGR{Ευάγγελος Στάμος}
%
%% -------------------------------------------------------------------
%% ΟΝΟΜΑΤΕΠΩΝΥΜΟ ΕΠΙΒΛΕΠΟΝΤΑ ΚΑΘΗΓΗΤΗ
% 
	\supervisor{Ioannis Papadopoulos}
%
%% -------------------------------------------------------------------
%% ΤΙΤΛΟΣ ΕΠΙΒΛΕΠΟΝΤΑ ΚΑΘΗΓΗΤΗ
%
	\supervisorTitle{Assistant Professor}
%
%% -------------------------------------------------------------------
%% ΕΠΙΒΛΕΠΩΝ/ΕΠΙΒΛΕΠΟΥΣΑ
%
% "Επιβλέπων" ή "Επιβλέπουσα", ανάλογα με το φύλο του 
% Επιβλέποντα Καθηγητή
	\supervisorMaleFemale{Supervisor}
%
%% -------------------------------------------------------------------
%% ΤΟΠΟΣ/ΜΗΝΑΣ/ΕΤΟΣ ΕΚΔΟΣΗΣ
%
	\thesisPlaceDate{Athens, February 2021}
%
%% -------------------------------------------------------------------
%% ΤΟΠΟΣ/ΜΗΝΑΣ/ΕΤΟΣ ΣΥΓΓΡΑΦΗΣ (Εμφανίζεται στη σελίδα των ευχαριστιών,
%% αν υπάρχει).
%
	\ackPlaceDate{Athens, February 2021}
%
%% -------------------------------------------------------------------
%% ΗΜΕΡΟΜΗΝΙΑ ΕΞΕΤΑΣΗΣ
%
	\examinationDate{26th February 2021}
%% -------------------------------------------------------------------
%% ΗΜΕΡΟΜΗΝΙΑ ΔΗΛΩΣΗΣ ΠΕΡΙ ΜΗ ΛΟΓΟΚΛΟΠΗΣ
%
	\declarationDate{26th February 2021}
%
%% -------------------------------------------------------------------
%% ΕΤΟΣ COPYRIGHT
%
	\copyrightYear{2021}
%
%% -------------------------------------------------------------------
%% ΟΝΟΜΑΤΕΠΩΝΥΜΟ 1ου ΕΞΕΤΑΣΤΗ
%
	\firstExaminer{Grigoris Papanikolaou}
%
%% -------------------------------------------------------------------
%% ΤΙΤΛΟΣ 1ου ΕΞΕΤΑΣΤΗ
%
	\firstExaminerTitle{Professor}
%
%% -------------------------------------------------------------------
%% ΟΝΟΜΑΤΕΠΩΝΥΜΟ 2ου ΕΞΕΤΑΣΤΗ
%
	\secondExaminer{Georgios Georgiou}
%
%% -------------------------------------------------------------------
%% ΤΙΤΛΟΣ 2ου ΕΞΕΤΑΣΤΗ
%
	\secondExaminerTitle{Associate Professor}
%%
%%
%%%%%%%%%%%%%%%%%%%%%%%%%%%%%%%%%%%%%%%%%%%%%%%%%%%%%%%%%%%%%%%%%%%%%%
%% THESIS COLORS: 
%%%%%%%%%%%%%%%%%%%%%%%%%%%%%%%%%%%%%%%%%%%%%%%%%%%%%%%%%%%%%%%%%%%%%%
%%
%% Χρώμα εξωφύλλου - κεφαλαίων
	\chaptercolor{gray!50!brown}
%%
%% Χρώμα παραρτημάτων
	\appendixcolor{brown!60!orange}
%%
%% Χρώμα υπερσυνδέσμων (αν έχει ενεργοποιηθεί η επιλογή "hyperref")
    \hyperlinkcolor{blue}
%%
%% Χρώμα τίτλου εργασίας στο εξώφυλλο (αν δεν έχει ενεργοποιηθεί 
%% η επιλογή "plain")
    \titlecolor{white}
%%
%% Χρώμα υποβάθρου (φόντου) τίτλου εργασίας στο εξώφυλλο (αν δεν έχει 
%% ενεργοποιηθεί η επιλογή "plain")
    \titlebackgroundcolor{gray!60!brown}  
%%
%%
%%%%%%%%%%%%%%%%%%%%%%%%%%%%%%%%%%%%%%%%%%%%%%%%%%%%%%%%%%%%%%%%%%%%%%
%% COVER PAGE IMAGE: 
%%%%%%%%%%%%%%%%%%%%%%%%%%%%%%%%%%%%%%%%%%%%%%%%%%%%%%%%%%%%%%%%%%%%%%
%%
%% Εικόνα εξωφύλλου (προαιρετική)
%% Στην περίπτωση κατά την οποία δεν είναι επιθυμητή η εισαγωγή εικόνας στο εξώφυλλο,
%% διαγράψτε την εντολή \coverpageimage, ή μετατρέψτε την σε σχόλιο (με "%")
%%
%% Σύνταξη:
%%          \coverpageimage{συντελεστής μεγέθυνσης}{όνομα αρχείου εικόνας [πλήρης διαδρομή]}
%%      ή
%%          \coverpageimage[tikz]{συντελεστής μεγέθυνσης}{εντολές TikZ}
%%          (στις εντολές μπορούν να περιλαμβάνονται και δηλώσεις \usetikzlibrary, κ.λπ.)
%%      
%% Παραδείγματα:
%%      - Χρήση εικόνας από το αρχείο "figures/rdf.png" με συντελεστή μεγέθυνσης 0.8:
%%          \coverpageimage{0.8}{figures/rdf.png}
%%      - Χρήση εικόνας TikZ με συντελεστή μεγέθυνσης 0.5:
%%          \coverpageimage[tikz]{0.5}{
%%              \draw[thick, gray] \foreach \x in {18,90,...,306} {
%%                  (\x:4) node{} -- (\x+72:4)
%%                  (\x:4) -- (\x:3) node{}
%%                  (\x:3) -- (\x+15:2) node{}
%%                  (\x:3) -- (\x-15:2) node{}
%%                  (\x+15:2) -- (\x+144-15:2)
%%                  (\x-15:2) -- (\x+144+15:2)
%%              };
%%          }
%%
     \coverpageimage{0.8}{figures/rdf.png}
%%
%%%%%%%%%%%%%%%%%%%%%%%%%%%%%%%%%%%%%%%%%%%%%%%%%%%%%%%%%%%%%%%%%%%%%%
%
% add custom hyphenation rules here
%\hyphenation{ο-ποί-α} 
%
%%%%
%
%
%%%%
\begin{document}

\maketitle

\beginfrontmatter

\renewcommand{\partname}{Part}
\renewcommand{\chaptername}{Chapter}
\captionsetup[figure]{labelfont={bf},name={Figure},labelsep=period}
\renewcommand{\contentsname}{Table of Contents}
\renewcommand{\listfigurename}{List of Figures}
\renewcommand{\listtablename}{List of Tables}
\renewcommand{\listillustrationname}{List of Images}
\renewcommand{\appendixname}{\Appendix}
\renewcommand{\bibname}{Bibliography}
\captionsetup[table]{labelfont={bf},name={Table},labelsep=period}

% Περίληψη
	\begin{abstracteng}
A peer-to-peer system is a set of autonomous computing nodes
(the peers) which cooperate in order to exchange data. The peers
in the peer-to-peer systems that are widely used today, rely on
simple keyword selection in order to search for data. The need for
richer facilities in exchanging data, as well as, the evolution of
the Semantic Web, led to the evolution of the schema-based
peer-to-peer systems. In those systems every node uses a schema to
organize the local data. So there are two ways in order for data
search to be feasible. The first but not so flexible way implies
that every node uses the same schema. The second way gives every
node the flexibility to choose a schema according with its needs,
but on the same time requires the existence of mapping rules in
order for queries to be replied. This way though, doesn't offer
automatic creation and dynamic renewal of the mapping rules which
would be essential for peer-to-peer systems.

This diploma thesis aims to the development of a schema-based
peer-to-peer system that allows a certain flexibility for schema
selection and on the same time enables query transformation
without the use of mapping rules. The peers use RDF schemas that
are subsets (views) of a big common schema called global schema.

   \begin{keywordseng}
    Peer-to-peer, Schema-based peer-to-peer, Semantic Web, RDF/S, RQL, Jxta
   \end{keywordseng}

\end{abstracteng}

% Αφιέρωση
	\thesisDedication{to my parents}
% Ευχαριστίες
	%%%%%%%%%%%%%%%%%%%%%%%%%%%%%%%%%%%%%%%%%%%%%%%%%%%%%%%%%%%%%%%%%
%%
%% use the starred version of the "acknowledgements" environment
%% to omit signatures from this section, e.g.:
%% \begin{acknowledgements*} ... \end{acknowledgements*}
%% 
%%%%%%%%%%%%%%%%%%%%%%%%%%%%%%%%%%%%%%%%%%%%%%%%%%%%%%%%%%%%%%%%%
\begin{acknowledgements}
Acknowledgements content . . .
\end{acknowledgements}

% Πίνακας Περιεχομένων
	\tableofcontents
% Κατάλογος Σχημάτων
	\listoffigures
% Κατάλογος Εικόνων
	\listofillustrations
% Κατάλογος Πινάκων
	\listoftables
% Πρόλογος
	\begin{preface}
Preface content . . .
\end{preface}

	
\beginmainmatter

%%%%%%%%%%%%%%%%%%%%%%%%%%%%%%%%%%%%%%%%%%%%%%%%%%%%%
%% INCLUDE YOUR CHAPTERS/SECTIONS HERE
%%

    
% Εισαγωγή
	\chapter{Introduction} 
Introduction content . . .

% Μέρη/Κεφάλαια

	\chapter{Chapter Name}
Chapter content . . .
\begin{tabbing}
1.<?x\=ml\= v\=ersion="1.0"?> \\
2.<rdf:RDFxmlns:rdf="http://www.w3.org/1999/02/22-rdf-syntax-ns\#" \\
3.\>\>\>xmlns:dc="http://purl.org/dc/elements/1.1/" \\
4.\>\>\>xmlns:exterms="http://www.example.org/terms/"> \\
5.\><rdf:Description
rdf:about="http://www.example.org/index.html"> \\
6.\>\><exterms:creation-date>August 16, 1999</exterms:creation-date> \\
7.\>\><dc:language>en</dc:language> \\
8.\>\><dc:creator rdf:resource="http://www.example.org/staffid/85740"/> \\
9.\></rdf:Description> \\
10.</rdf:RDF> \\
\end{tabbing}
}

    \part{Epilogue}
	\include{body_matter/chap9}
% Παραρτήματα
	%\appendices
% Βιβλιογραφία - Αναφορές
	\bibliography{references}
% Συντομογραφίες - Αρκτικόλεξα - Ακρωνύμια
	\includeabbreviations{back_matter/abbreviations}
	
%%%%%%%%%%%%%%%%%%%%%%%%%%%%%%%%%%%%%%%%%%%%%%%%%%%%
% Ευρετήριο Όρων
	\printindices
%
%%%%%%%%%%%%%%%%%%
%%%%%%%%%%%%%%%%%%

%% Δημιουργία ετικετών CD:

	\definecdlabeloffsets{0}{-0.65}{0}{0.55} % upper label x offset [cm] (default=0) /  upper label y offset [cm] (default=0) /  lower label x offset [cm] (default=0) /  lower  label y offset [cm] (default=0) -- For Q-Connect KF01579 labels use the following offset values: {0}{-0.65}{0}{0.55}

	\createcdlabel{Πρότυπο Σύστημα Ομότιμων \\ Κόμβων Βασισμένο σε Σχήματα \en{RDF}}{Κωνσταντίνος Δ. Δημητρίου}{ΟΚΤΩΒΡΙΟΣ}{2020}{8} % τίτλος διπλωματικής / όνομα συγγραφέα / μήνας / έτος / εύρος περιοχής τίτλου σε cm (προτεινόμενη τιμή: 8) 

%%σ
%% Δημιουργία εξωφύλλου θήκης CD:

	\createcdcover{Πρότυπο Σύστημα Ομότιμων \\ Κόμβων Βασισμένο σε Σχήματα \en{RDF}}{Στάμος Φ. Ευάγγελος}{ΟΚΤΩΒΡΙΟΣ}{2020}{10} % τίτλος πτυχιακής / όνομα συγγραφέα / μήνας / έτος / εύρος περιοχής τίτλου σε cm (προτεινόμενη τιμή: 10) 

%%
%
\end{document}

%%%%%%%%%%%%%%%%%%%%%%%%%%%%%%%%%%%%%%%%%%%%%%%%%%%%
