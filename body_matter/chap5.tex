\chapter{Υλοποίηση}
\InitialCharacter{Σ}το κεφάλαιο αυτό περιγράφεται η υλοποίηση του συστήματος, με βάση τη μελέτη που παρουσιάστηκε στο προηγούμενο κεφάλαιο. Αρχικά παρουσιάζεται η πλατφόρμα και τα προγραμματιστικά εργαλεία που χρησιμοποιήθηκαν. Στη συνέχεια δίνονται οι λεπτομέρειες υλοποίησης για τους βασικούς αλγορίθμους του συστήματος καθώς και η δομή του κώδικα.

\section{Λεπτομέρειες υλοποίησης}
Στην ενότητα αυτή παρουσιάζονται οι βασικοί αλγόριθμοι που
αναπτύχθηκαν καθώς και λεπτομέρειες σχετικά με την υλοποίηση της
επικοινωνίας των κόμβων.

\subsection{Αλγόριθμοι}

\subsubsection{Αλγόριθμος εισαγωγής δεδομένων}
Όταν ένας κόμβος εισέρχεται για πρώτη φορά στο σύστημα, αρχικά
δημιουργεί το σχήμα που θέλει χρησιμοποιώντας το \en{RDFSculpt}.
Στη συνέχεια................

\noindent\textbf{Παράδειγμα} \\

Έστω ότι ο κόμβος έχει επιλέξει να συμμετέχει στο σύστημα με το \en{RDF} σχήμα που φαίνεται
στο Σχήμα. Έστω επίσης ότι από το \en{SQL} ερώτημα που έχει κάνει στη σχεσιακη
βάση, έχει προκύψει η όψη που φαίνεται στον Πίνακα. Για τις ανάγκες του παραδείγματος θεωρούμε
ότι η όψη αυτή περιέχει μόνο μία εγγραφή.

...........................

\section{Περιγραφή κλάσεων}
Στην ενότητα αυτή δίνεται μια σύντομη περιγραφή των κλάσεων,
των πεδίων και των μεθόδων που τις απαρτίζουν.

\subsection{\en{public class FirstUi}}
\noindent Η κλάση αυτή κατασκευάζει την οθόνη εισαγωγής του χρήστη στο σύστημα.\\

\noindent\textbf{Πεδία}

\begin{itemize}
\item\src{private GridBagLayout blayout} \\
Το \en{layout} για όλα τα \en{Panel}.
\item\src{private GridBagConstraints con} \\
Τα \en{constraints} για το \en{layout}.
\item\src{private Icon arrowR} \\
Εικονίδιο για το κουμπί \en{Next}.
\end{itemize}

\noindent\textbf{Μέθοδοι}

\begin{itemize}
\item\src{public FirstUi()}\\
Ο κατασκευαστής της κλάσης ο οποίος καλεί την \en{createEntryFrame()}.
\item\src{private void createEntryFrame()}\\
Μέθοδος που κατασκευάζει το en{frame}.
\end{itemize}