\chapter{Περιγραφή θέματος}
\InitialCharacter{Σ}το κεφάλαιο αυτό αρχικά γίνεται μια περιγραφή των συστημάτων ομότιμων κόμβων που είναι βασισμένα σε σχήματα \en{(schema-based peer-to-peer systems)}. Στη συνέχεια περιγράφονται τρία βασικά συστήματα που ανήκουν σε αυτή την κατηγορία, καθώς και ένα σύστημα για τη διαχείρηση \en{RDF} σχημάτων, και τέλος αναλύεται ο στόχος
της παρούσας εργασίας.

\section{Σχετικές εργασίες}
Οι βάσεις δεδομένων εισήγαγαν ένα τρόπο αποθήκευσης και ανάκτησης
των δεδομένων που βασιζόταν στο σχήμα \cite{elli05}. Τα
πρώτα συστήματα ομότιμων κόμβων που περιγράψαμε στην Υποενότητα
2.1.2 έδιναν μεγάλη σημασία στην αρχιτεκτονική του συστήματος και
την δρομολόγηση των ερωτήσεων και λιγότερη στον τρόπο
αναπαράστασης και τις δυνατότητες αναζήτησης. Η αναζήτηση σε αυτά
τα συστήματα ομότιμων κόμβων γίνεται με βάση προκαθορισμένα
χαρακτηριστικά - δείκτες, ή με προσπάθεια αντιστοίχισης μιας λέξης
κλειδί. 

Η ανάγκη λοιπόν για πιο εκφραστικές λειτουργίες οδήγησε στα
συστήματα ομότιμων κόμβων τα οποία είναι βασισμένα σε σχήματα
(\en{schema based peer-to-peer systems})\indexEN{schema based peer-to-peer systems}. Πρόκειται για ομότιμες
υποδομές διαχείρισης δεδομένων που όμως διατηρούν όλα τα
χαρακτηριστικά των συστημάτων ομότιμων κόμβων.
............................